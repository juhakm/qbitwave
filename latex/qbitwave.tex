\section{Interpretation Space of a Bitstring Universe}

\subsection{Bitstring as Invariant Substrate}

Let
\[
B \in \{0,1\}^N
\]
be a finite binary string of length \( N \). This bitstring represents the fundamental information substrate of a universe, invariant across interpretations.

\subsection{Interpretation Function}

We define an \emph{interpretation} (or observer decoding scheme) as a function
\[
\mathcal{I}: \{0,1\}^N \rightarrow \mathcal{H}
\]
where \( \mathcal{H} \) is a Hilbert space of complex-valued vectors, i.e., possible wavefunctions.

Each interpretation consists of:
\begin{itemize}
  \item \textbf{Block size} \( k \in \mathbb{Z}^+ \): how many bits to group per amplitude.
  \item \textbf{Parsing rule} \( P \): maps bit blocks into complex amplitudes.
  \item \textbf{Domain mapping} \( D \): assigns amplitudes to positions (e.g., spatial or phase).
  \item \textbf{Normalization} \( N \): scales wavefunction to satisfy norm constraints.
\end{itemize}

Thus, the interpretation becomes:
\[
\mathcal{I}_{k,P,D,N}: B \mapsto \Psi_B
\]
with output \( \Psi_B \in \mathbb{C}^M \), where \( M = N / k \).

\subsection{Observer Class and Interpretation Space}

Let \( \mathcal{O} \) denote the space of all interpretations an observer might apply:
\[
\mathcal{O} = \left\{ \mathcal{I}_{k,P,D,N} \,\middle|\, k, P, D, N \text{ admissible} \right\}
\]
Each \( \mathcal{I} \in \mathcal{O} \) represents a unique physical reality consistent with the same bitstring \( B \).

\subsection{Equivalence of Interpretations}

We define observational equivalence as:
\[
\mathcal{I}_1 \sim \mathcal{I}_2 \iff \left| \mathcal{I}_1(B) \right|^2 = \left| \mathcal{I}_2(B) \right|^2
\]
This identifies equivalence classes over \( \mathcal{O} \), where different decoding rules yield indistinguishable outcomes.

\subsection{Emergence and Optimality}

Define a scoring function to quantify structure:
\[
S(\mathcal{I}, B) = H(\Psi_B)
\]
where \( H \) is a structure or entropy-based metric. Then the \emph{emergent interpretation} is:
\[
\mathcal{I}^* = \arg\max_{\mathcal{I} \in \mathcal{O}} S(\mathcal{I}, B)
\]

This formulation justifies why certain block sizes or structures \emph{emerge} naturally: they maximize the informational yield from \( B \).

\subsection{Summary of Symbols}

\begin{center}
\begin{tabular}{ll}
Symbol & Description \\
\hline
\( B \) & Raw bitstring (substrate) \\
\( \mathcal{I} \) & Interpretation function (observer's decoding) \\
\( \mathcal{O} \) & Set of all possible interpretations \\
\( \Psi_B \) & Wavefunction derived from \( B \) via \( \mathcal{I} \) \\
\( H \) & Entropy or structure score \\
\( \mathcal{I}^* \) & Optimal interpretation maximizing \( H \) \\
\( \sim \) & Observational equivalence relation \\
\end{tabular}
\end{center}
